\documentclass[fleqn,dvipdfmx,a4j]{scrartcl}
\usepackage{ascmac,multicol,tikz,amsmath,amssymb,fancybox}
\usepackage[normalem]{ulem}
\usepackage[title=small]{phfnote}
\usetikzlibrary{intersections,calc,arrows.meta}
\title{国語前期小論文課題(課題2)}
\date{2025/09/29}
\author{
  E3(12)木内 大
}
\renewcommand{\footnotesize}{\small}
\begin{document}
\maketitle
\vspace{-15pt}
    \quad 本文において、ひまわり学生運動の占拠を継続するか否かの多数決で、占拠を続けるべき主張をする学生が主張が通らずとも「ありがとう」と言ったこと、
    フィンランドの映画「100,000年後の安全」においてフィンランド政府が情報公開を躊躇わなかったこと、
    「特定秘密保護法」反対デモのためのCM動画が楽しみ・ユーモアに溢れていたことが挙げられていた。これらの具体例から、
    「意見が通らなかった少数派が多数派に感謝の気持ちが持てること」、「情報公開に対する誠実さがあること」、「政治に参加する楽しさがあること」という、
    3つのことが高橋さんの「正しい民主主義」のための条件であった。\\

    \quad これに加えて私は、「広い視野・立場で意見・物事を観察できること」を「正しい民主主義」のための条件に付け加えるべき事柄として挙げる。\\
    \quad 
    民主主義では、多様な立場や価値観を持つ人々が共存しているため、1つ・または少ない観点だけで物事を見ると、
    対立や分断が深まってしまったことが発生しうるからである。
    例えば古代アテネでは、民主制が行われていたにはいたが、男子市民にのみ参政権が与えられた\cite{history}。日本でも、選挙権が直接国税15円以上を納めた満25歳以上の男性のみに与えられた時代が存在した。
    このようなことは、王政と比較すれば、市民(一部ではあるが)の意見が政治に反映されやすい制度ではあったが、「真の平等」と言える状態ではなかった。
    欠けている立場の意見は、他の政治参加者の耳には入りすらしないためにこの民主主義は部分的で不完全になると言える。
    このような歴史的な事柄から考えると、1つ・または少ない観点からだけで物事を見ることは、良くないことが分かる。\\
    \quad 
    その他に、人は「確証バイアス」という、自らの見たいもの・信じたいものを信じるという心理的特性を有していることや、現代では、「エコーチェンバー」という、
    SNSを利用する際に自分と似た興味関心をもつユーザーをフォローする結果、意見をSNSで発信すると自分と似た意見が返ってくるという現象\cite{echo-chamber}もあり、
    多くの視点から考えることの重要性が増していると考える。\\
    \quad 
    意見の違いは、生活スタイル、経済状況、宗教、思想など様々な要因で生じるため、1つの分野に限らず広い視野を持つことで、
    自分と異なる意見の背景や根拠を理解しやすくなり、互いを尊重することや建設的な議論が可能になるということもあると考える。
    \\


    \quad 
    次に、実現された際の効果についてである。\\
    \quad 
    まず、昨今はSNSが普及していることがあり、以前と比べて手軽に意見が発信できる。
    しかしながら、手軽に発信できるということとは表裏一体に、
    フェイクニュースや根拠がはっきりしない情報が拡散されることがあるように感じる。また、先から述べている「エコーチェンバー」、
    さらにはAIの発展で、ありもしない画像
    \footnote{例えば、2022年9月の台風15号による豪雨の際、AIに生成させた画像をドローンで撮影した静岡県内の災害の様子とする投稿があった\cite{fake-news}。}
    が拡散されることも起きている。
    このような状況のため、1つの情報源に頼りすぎていると、偏った意見を持ってしまう可能性がある。これを防ぐことができるのが、効果として挙げられると考える。\\
    \quad 
    第二に、高橋さんが先の条件で「情報公開に対する誠実さがあること」を挙げていることを加味して、
    国民一人ひとりが偏らない情報収集を行なった上で柔軟な思考を行うことができることになり、政治の質も高められると考える。
    国の政策も、災害や経済、環境問題などが関わって決定されていることがあり、
    このような側面からも、より広い視野で物事を観察できることで事柄の理解を深められるため、重要になってくると考えらえる。\\

\begin{thebibliography}{9}

\bibitem{history}
  『最新世界史図説タペストリー二十二訂版』帝国書院
  \newblock (2024/02/25発行)

\bibitem{echo-chamber}
  『総務省|令和元年版 情報通信白書|インターネット上での情報流通の特徴と言われているもの』総務省
  \newblock \url{https://www.soumu.go.jp/johotsusintokei/whitepaper/ja/r01/html/nd114210.html}
  \newblock (2025/09/10閲覧)

\bibitem{fake-news}
  『SNS拡散の災害デマやフェイク画像 “AIが生成した偽画像”も - NHK』NHK
  \newblock \url{https://www3.nhk.or.jp/news/special/saigai/select-news/20220928_01.html}
  \newblock (2025/09/09閲覧)

\end{thebibliography}

\end{document}