\documentclass[fleqn,dvipdfmx]{jsarticle}
\usepackage{ascmac,multicol,ulem,tikz,amsmath,amssymb,fancybox,}
\usepackage[inline]{enumitem}
\usepackage{multicol}
\usepackage[colorlinks=false, pdfborder={0 0 0}]{hyperref} 
% \usepackage[title=default]{phfnote}
\usetikzlibrary{intersections,calc,arrows.meta}
\title{産業革命と鉄道の発展}
\date{2025/09/24}
\author{
  E3(12)木内 大
}
\begin{document}
% タイトルを中央揃え
\begin{center}
  {\LARGE 産業革命と鉄道の発展}\\[2em]  % 2em は余白の調整
\end{center}

% 氏名を右揃え
\begin{flushright}
3E12 木内 大
\end{flushright}

% \tableofcontents
\section{このテーマを選択した理由}
これまで高専では、(電気が中心ではあるものの)これまでの歴史の上にある工業について学んできており、その発展までの歴史的背景をより深く知ることが大切だと思った。
そのような中で自分自身が鉄道での旅行が好きであるために、それの発展までの過程、及び産業革命へ及ぼしたことを調査することにした。
\section{産業革命以前の鉄道事情}
\subsection{鉄道の起源}
もともと鉄道の起源は、古代ローマ時代にまで遡ることができる。\\
当時は馬車での輸送が主流であったが、車輪が地面の上を転がると、地面に「轍」という跡がついたり、車輪が地面にめり込んでしまうなどで、輸送効率が悪かった。\\
これを解決するために、石を敷き詰めて車輪がめり込まないようにした。
古代ローマでは、石畳の道路が建設され、これが後の鉄道の基礎となった。\\
しかし、路面の全てを覆うためには材料が多く必要で、手間もかかる。車輪だけが通る部分だけを固くすることができれば、材料も手間も省ける。
このことから、車輪がレールの上を転がる仕組みが考えられ、16世紀ごろからヨーロッパ各地で鉱山の坑道内に木製のレールが敷設されるようになった。\\
また、1760年代からは、鉄製のレールが使用されるようになり、耐久性が向上した。\\
中世ヨーロッパでは、鉱山での鉱石輸送のために、木製のレールを用いたトロッコが使用されるようになった。\\
これが、鉄道の原型とされている。
\subsection{馬車鉄道}
鉄道のメリットとして、鉄車輪が転がることで、摩擦が小さくなり、同じ力でより重い物を運ぶことができるという点がある。\\
そのため、これを馬車に応用して、鉄車輪付きの車両を載せ、人よりも多くの力を出せる馬に牽引させることで、より多くの物資を運ぶことができるようになった。\\
馬が貨車を牽引する馬車鉄道は、17世紀ごろまでヨーロッパ各地に存在した。\\
明確な記録があるものとしては、1803年に開通した、イギリスのサリー鉄道がある。\\
この鉄道は、石炭や肥料、農作物、工業製品などを輸送する貨物鉄道で、世界で最初の馬車鉄道とされている。\\
その後、1804年にはスウォンジー・アンド・マンブルズ鉄道が開業した。元々は採石場から港まで石灰石等の鉱石を運ぶ馬車鉄道であったが、1807年に旅客輸送を開始した。\\
そのため、世界初の旅客輸送を行なった馬車鉄道と言われている.

\section{産業革命とそれによる変化}
\subsection{産業革命の大まかな概要}
産業革命は、18世紀のイギリスから始まった。
初めは木綿工業分野において、工場制手工業から機械工業へ変化したことから始まる。\\
その後、鉄鋼業や機械工業などの分野においても同様の変化が起こり、産業全体が機械化されていき、
動力革命や交通革命、通信革命といった変化が起きたことを産業革命という。
\subsection{蒸気機関の発明}
蒸気機関は、イギリスのトーマスニューコメンが1712年に考案した。
ただしこれは問題があり、シリンダーが冷えてから温めるための燃料消費が多いこと、回転運動ではなく往復運動で汎用性が低いことがあった。(下図左側)\\
これをJ.ワットが1765年に改良し、シリンダーの冷却を防ぐための外部冷却装置を追加した。
これにより、燃料消費が減り、蒸気だけで動くようになり、効率も向上した。
また、1784年、同じくJ.ワットが往復運動を回転運動に変えることに成功し、蒸気機関の汎用性を高めた。(下図右側)\\
これらのことは、鉱山の排水や工場の製粉などの作業が自動化され、生産効率を高まった。\\
そして、蒸気機関は輸送機関の動力としても使用され、その結果誕生したのが蒸気自動車、蒸気船、そして蒸気機関車である。
\begin{figure}[htbp]
\centering
\includegraphics[height=70mm, width=130mm]{History_Jokikikan.png}
  \caption{蒸気機関の基本的な構造}
  \label{fig:steam_engine}
\end{figure}
\subsection{蒸気機関車の発明}
蒸気機関車そのものは、1804年にリチャード・トレビシックによって発明された「ペナダレン号」が最初である。この機関車は、10トンの鉄と70人が乗車した5両の客車を牽引し、平均速度3.9km/hで走行した。しかしこれは、課題が多く本格的な鉄道では実用化されなかった。\\
実用的な蒸気機関車は、ジョージ・スティーブンソンが1814年に、鉱山の鉄道むけに発明したものである。スティーヴンソンは、この後に開業した鉄道の建設や、導入された蒸気機関車にも携わり、「鉄道の父」や「蒸気機関車の父」と呼ばれている。\\
特に1825年に開業したストックトン・アンド・ダーリントン鉄道では、スティーブンソンが設計した蒸気機関車「ロコモーション号」が使用され、総重量90トンの列車を牽引し、最高速度18km/hで走行した。ストックトン・アンド・ダーリントン鉄道は蒸気機関車が列車を牽引した世界初の公共鉄道といわれている。
\section{蒸気機関車による鉄道の開業と発展}
\subsection{世界初の公共交通鉄道の開業}
1830年、世界初の営業鉄道のリバプール・アンド・マンチェスター鉄道が開業した。\\
海に面した港町であるリバプールから、工業地帯のマンチェスターまでを結ぶ都市間鉄道であった。この鉄道は、開業時から蒸気機関車が牽引する列車が、発着時刻を予め決めた時刻表に基づいて走り、現在の営業鉄道の基礎を築いたため、「世界初の本格的な鉄道」や「世界初の営業鉄道」と呼ばれている。\\
成り立ちとしては、工業製品やその原料などを輸送する貨物鉄道として計画されたが、開業後は旅客輸送も行うようになり、営業的に成功を収めた。\\
この時に最初に導入された蒸気機関車「ロケット号」は、ジョージ・スティーブンソンとその息子ロバート・スティーブンソンによって設計されたもので、最高速度は48km/hに達した。馬車鉄道よりも多くの人や物を一度に速く運ぶことができるようになり、人々はその有益性に注目した。
\subsection{鉄道開業の影響}
マンチェスターは工業都市として発展したが、鉄道の開業は、どれほど影響があったのだろうか。\\
以下は、マンチェスター市の人口推移\cite{population1}である。
\begin{figure}[htbp]
\centering
\includegraphics[height=60mm, width=120mm]{Manchester_Population.png}
  \caption{19世紀のマンチェスター市の人口推移}
  \label{fig:steam_engine}
\end{figure}\\
産業革命以降の1801年時点で、人口が多かったマンチェスター市\footnote{グラフの元となった文献によれば、産業革命前の1081年時点の人口は3000人である。}ではあるが、鉄道開業後の1831年から1851年にかけては、それより前の1801年から1831年にかけてと比較して、人口増加が急激である。
港まで(馬車や徒歩と比較して)すぐに行くことができるようになり、これがこの市の人口が増加した一因と考えられる。\\
一方以下は、リバプール港の綿花の輸入量、輸出量、販売量の推移\cite{cotton1}である。(この資料の綿花の単位は「パック」、即ち箱数である。)
\begin{figure}[htbp]
\centering
\includegraphics[height=60mm, width=120mm]{Liverpool_Cotton.png}
  \caption{19世紀のリバプールでの綿花の輸入量、輸出量、販売量の推移}
  \label{fig:steam_engine}
\end{figure}\\
1830年以降の綿花の輸入についてはそれより前の増加よりも急激である。
鉄道の開業により、綿花の輸送が容易になり、これが綿花の輸入・輸出の増加に寄与したと考えられる。
また、販売量が上昇していることから、綿織物の需要が増加傾向であり、労働者がマンチェスターへ移住したことが人口増加の他の理由の1つとしても考えられ、
鉄道だけが人口増加の理由ではないと言えてしまうが、鉄道開通で生産量増加をさせ、産業革命の影響をより大きくしたのではないかと考えられる。
\subsection{鉄道狂時代へ}
このように、鉄道の有益性が認識されると、投資家らは「鉄道が儲かる」と考え、鉄道への投資に夢中になり、イギリス各地で鉄道建設が進んだ。その結果1830年代から1840年代にかけて「鉄道狂時代」と呼ばれる鉄道建設ブームが起こった。\\
この時期には、多くの鉄道会社が設立され、鉄道網が急速に拡大した。\\
鉄道の発展は、産業革命をさらに促進し、都市の発展や経済の成長に大きく寄与した。\\
また、この動きはイギリスにとどまらず、ヨーロッパ大陸やアメリカ合衆国など他の地域にも広がり、世界的な鉄道網の発展につながった。
% \section{余談}
% \subsection{レールの幅について}
% 鉄道のレールの幅は、国や地域によって異なる。\\
% 日本の場合、多くの在来線は狭軌(1067mm)、新幹線や一部の在来線は標準軌(1435mm)が使用されている。\\
% 規格を揃えておけば、直通運転ができたり、メンテナンスで部品の融通が利いたりして、メリットがあると僕は考え、そこはいいと思う。
% 実際にイギリスでも、統一された規格の標準軌の線路で開通している。
% しかしどうして、これら規格化された線路幅は中途半端な数値の幅なのだろうか、またイギリスではどの幅を使用していたのだろうかと疑問に思った。
\section{おわりに}
鉄道は、産業革命において直接的に影響を与えたわけではないが、産業革命の機械化により増加した生産量を、さらに増加させる起爆剤となり、産業と鉄道との間には相互関係があるのではないだろうか。
\begin{thebibliography}{9}

\bibitem{jokikikan}
『蒸気機関』世界史の窓
\newblock \url{https://www.y-history.net/appendix/wh1101-025.html}
\newblock (2025年8月22日閲覧)

\bibitem{jokikikan2}
『第2回\quad 蒸気機関の普及\quad ~ワットの蒸気機関~』株式会社IRS\quad
\newblock \url{http://www.irs-japan.com/?p=3849}\quad
\newblock (2025年8月22日閲覧)

\bibitem{jokikikan3}
『まなびのずかん\quad 世界と日本の鉄道史』株式会社技術評論社\quad
\newblock 川辺 \quad 謙一\quad
\newblock (2022年10月27日初版発行)\quad

\bibitem{population1}
『Public Intelligence A20 Manchester's population over time』
Manchester City Council\quad
\newblock \url{https://www.manchester.gov.uk/download/downloads/id/25393/a20_1086-2016_manchester_population.pdf}
\newblock (2025年9月7日閲覧)

\bibitem{cotton1}
『THE LIVERPOOL COTTON BROKERS ASSOCIATION
AND THE CROWNING OF KING COTTON, 1811-1900:
EXAMINING THE ROLE OF A PRIVATE ORDER
INSTITUTION IN GLOBAL TRADE』
\newblock QUCEH WORKING PAPER SERIES\quad
Michael Aldous \& Christopher Coyle\quad
\newblock (2020年10月)

\end{thebibliography}
\end{document}