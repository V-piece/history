\documentclass[dvipdfmx, a4paper]{jsarticle}
\usepackage[dvipdfmx]{graphicx}
\usepackage{ascmac,multicol,ulem,tikz,amsmath,amssymb,fancybox,}
\usepackage[inline]{enumitem}
\usepackage{multicol}
\usepackage{titlesec}

\usepackage{pdfpages}
\usepackage[outputdir=/Users/kiuchidai/Documents/latex_out]{minted}
\usepackage{url}
\usepackage{fancyhdr} % ヘッダーをカスタマイズするパッケージ

% --- ヘッダー設定ここから ---
\pagestyle{fancy}
\lhead{\textbf{国語小論文(課題2)}}      % 左上に表示したい文字(著者名など)
\rhead{E3(12)木内 大 - 2026/02/16}   % 右上に現在のセクション名とかタイトルとかを表示
\cfoot{\thepage}         % フッター中央にページ番号
\renewcommand{\headrulewidth}{0.4pt} % ヘッダーの下に線を引く(0ptで消える)
% --- ヘッダー設定ここまで ---
% お好みで設定(枠線や行番号など)
\setminted{
    frame=lines,
    framesep=2mm,
    baselinestretch=1.2,
    fontsize=\small,
    linenos
}
%ここまでソースコードの表示に関する設定
\author{E3(12)木内 大}
\title{国語小論文(課題2)}
\date{2026/02/16}

\begin{document}
坂口さんの考える「真の美」は, \textbf{「必要」であるから生まれたものに備わり}, 
一方で美を意識して成されたところからは生まれない,というものであった.
\par
「必要」から「真の美」が生まれる場所としては, 小菅刑務所,佃島のドライアイス工場, 
そして駆逐艦がこれに当てはまる具体例として挙げられていた. 
これらの建物 または建造物は, 
美的修飾がなく(美しく加工するということも念頭に置かれておらず)
必要に応じた設備のみで形態が出来上がっている点, 
何も補わなくても,まっすぐに心を感動させ,郷愁を感じさせる点, 
我々の実際の生活が実質として表現されている点
で共通していた.
\par
逆に, 美を意識して成されたものとしては, 例えば、法隆寺・平等院といった
歴史的建築物や文学のことが挙げられていた. 
これらは, 歴史を前提として(例えば法隆寺なら「あの聖徳太子が作った」のような), 
そして何か納得せねばならない理由があり, 
この補う事項をもってしてやっと認識できるような美しさであった.
文学も, 美しく見せるための1行があってはならず, 書く必要があることのみで構成されており, 意識して美しくある必要はない, ということであった.
\par
従って, 必要によって生まれたものの発する美こそ, 「真の美」であり, それ以上の余計な装飾(見方によっては,見た目の誤魔化しみたいな,無駄な物とでも言うのか)は要らない, 
というのが彼の考える「真の美」である. 
\\\par
ここまで坂口さんの考える「真の美」について述べてきた. そして,これを基に自分が「必要性から生じた美」を感じる部分を考える. 
\par
「必要性から生じた美」を感じるものとして, まずはピクトグラムを挙げたい.
ピクトグラムは, 例えばトイレのマークや非常口などが挙げられる. 
これらは, 例えばトイレのマークであれば、男性用は男性のシルエット、女性用は女性のシルエットであるなど, その意味を伝えるために必要な要素以外は一切ない. 
また、非常口のマークであれば、逃げる人を示すマークの他は一切ない.
このように, 余計な装飾がなく, 伝えるべき意味を伝えるために必要な要素のみで構成されている点で,自分はピクトグラムから「必要性から生じた美」を感じることができる.\par
少し蛇足の話だが, ピクトグラムというものについては, 装飾をしなさすぎて訳がわからなくなるケースがあると思う. 
マークをよりシンプルにと突き詰めすぎてしまうと, 例えばトイレのマークで、男性用も女性用も区別が付かない状況になってしまうなど, 伝えるべき意味が伝わらなくなってしまうこともある. 
余計な装飾をしないことで発される美が「必要性から生じた美」だが, 必要な要素がないと伝えるべき意味が伝わらなくなってしまうこともあるため, 
余計な装飾をしないことと必要な要素を残すことのバランスが大事であると感じる.
\par
続いて, 東海道新幹線の車両N700系にも「必要性から生じた美」を感じる. \par
この車両は, 最高速度300km/hで走るために, 前面は特徴的な形をしているし, 
車体傾斜装置という装置でカーブでも速く走れるようになっている.
一方, 車体の色は白と青のみの非常にシンプルなものである. また車内でも, 個室があるのかと言えばそうではなく, 
全てが座席であるし,飛行機みたいに食事や飲み物のサービスがあるわけでもない. \par
結局のところ, 余計な装飾,またはサービスがなく, 
速く走る為に必要な設備以外は必要最低限構成されている為に東海道新幹線から「必要性から生じた美」を感じることができると思った. \par
そして, 「必要から生じた美」を感じるものとして最後に挙げたいのは, である. \par
他に物体ではなく, プログラミングにも「必要から生じた美」を感じることができると思う.
プログラミングは, 例えばPythonやC++などのプログラミング言語を用いて, コンピュータに命令を与えるためのコードを書くことを指す.
言語によって, 書き方は様々である. 同じ条件分岐でも, 例えばPythonであれば、\textbf{if (条件の式):(満たした時に実行)}, C++であれば\textbf{if (条件の式) { 実行するコード }},ある.
なんであれ, これらは元の英語の表現(If (条件の式),(満たした時に実行).)から段々省略してできており, 最低限必要な要素だけで構成されている. 
目で追いかけてみると, 書くにあたって余計な要素(例えば動詞)がなく, 必要な要素だけで構成されている点で, プログラミングからも「必要から生じた美」を感じることができると思う.(実際はプログラムは, コーディングを楽にしたり容量削減や可読性のために, このような書き方になったとは思うが, 結果的に必要な要素だけで構成されている点で「必要から生じた美」を感じることができると思う.)
\\\par
坂口さんの考える「真の美」について, 僕は特にシンプル・イズ・ベストの考え方に近いのではないだろうかと考える. 
余計な装飾がなく, 必要な要素だけで構成されているとは, まさにシンプル・イズ・ベストの考え方である.
そして「真の美」を発するものを単純に捉えることができるのは, 余計な装飾がないからであるし, 
伝えるべき意味を伝えるために必要な要素のみで構成されているからこそ, まっすぐに心を感動させることができるのだと思う.
\end{document}