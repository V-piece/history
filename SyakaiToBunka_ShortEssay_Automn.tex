\documentclass[fleqn,dvipdfmx,twocolumn]{jsarticle}
\usepackage{ascmac,multicol,ulem,tikz,amsmath,amssymb,fancybox,}
\usepackage[inline]{enumitem}
\usepackage{multicol}
\usepackage[colorlinks=false, pdfborder={0 0 0}]{hyperref} 
% \usepackage[title=default]{phfnote}
\usetikzlibrary{intersections,calc,arrows.meta}
\title{社会と文化レポート}
\date{ }
\author{
  E3(12)木内 大
}
\begin{document}
  \maketitle
  \section{はじめに}
    今回自分が選んだ課題は\textbf{場所とはいかなるものか授業の内容を踏まえて考える}である.私は場所とは以下のように考える.
    \begin{enumerate}
      \item 現実または空想に存在すること
      \item ただの物体ではなく空間であること.
      \item 記憶の上で存在することができること
      \item 古来から現在まで引き継がれている文化/自然等があってもなくても良いが,場所の文化や自然は変化し続けていること.
    \end{enumerate}

    \section{現実または空想に存在する とは}
      まず少なくとも,場所が現実に存在する場合は緯度経度や住所で表すことができる.例えば,北緯XX度XX分,東経XX度XX分という緯度経度で与えられているし,弊学住所が日本の静岡県沼津市大岡3600であるように,世界では場所に住所がある.この情報と地図によって場所が特定でき,訪れることができる.これは間違いなく,場所であるための要素と言えると思う.\\
      \quad あるいは,記憶によって存在していないものが存在してしまっていることがあるが,ある意味ではこれも場所が存在している状況と言えるのではないだろうか.授業で出てきた鯖街道のことや,相馬の古内裏のこともそうだが,空想で存在しない場所/物事が存在するかのように人々に広がっている.
      何なら現代では,アニメなどの聖地やモデルになった場所というのもあるわけだが,これも実際の登場人物が現実世界のそこにいるわけではない\footnote{また,地名/店名などが違い,場所としての名前が現実の存在と噛み合わないこともある.一例として挙げる.\cite{yurucha}}.
      しかし「ここでOOがXXした」のような空想によって現実に存在しているかのように思えるというのは,それ即ち脳の中としては場所があると言えるのではないかというのが考えである.

    \section{ただの物体ではなく空間である とは}
      思い出の場所が存在するとして,その場所にゆかりがある物体から場所までを思い出すことが難しい場合がある.授業では,思い出の場所の土では場所までは思い出せないことが引き合いに出され,私は実際共感した.従って,要素に追加するべきと考えた.

    \section{記憶の上で存在することができる とは}
      これより前の項目では,座標や住所などの絶対的な指標における存在のことを主としてきたが,それ以外の方法でも場所は存在すると思う.\\
      \quad 例えば,合掌造りと聞くとそれは大抵飛騨地方から富山県の周辺(もっと絞り白川郷となることもあるかも)となる.これはまさに場所のセンスによるものであり,特徴的な伝統が続けられているからこそ,この発想に至るのだと考える.
      また,ディズニフィケーション\footnote{資本を相当投入した場合,想像の具現化がされた状態.}によって伝統ではなくとも写真を見るなどして場所を絞り込めることがある.ここまでのことは,具体的であり記憶にある限りは絞り込めることだと思う.\\
      \quad 一方で没場所性\footnote{資本の投入で場所のセンスが隠れてしまっている状態.}が発生している状態,即ち全国各地にチェーン店が存在するということはよくある.しかし,それについても記憶が共有されている場合であれば,行き先を1箇所に限定ができる.\\
      \quad 例えば,仮に私がららぽーとに行くということを弊学寮で発言したならば,それ即ちその人はららぽーと沼津へ行くのだということになる.磐田や海老名ということには中々ならない.
      これは,沼津市にはららぽーとが存在しているということをお互いに認識しあっているからこそ通じるものである.仮にもこれを,静岡県西部や神奈川県などの,沼津の周辺地域に住んでいない人に言った場合はその地域に対しての回答が返ってくると思う.\\
      \quad 従って,別の回の授業(記憶のこと)に寄ってしまうが,結局は\textbf{記憶に存在しているか,そしてメンタルマップにおける存在,即ち記憶で名称と紐づけられているか}が場所であるかないかの鍵になってくると考える.\\

    \section{古来から現在まで引き継がれている文化/自然等があってもなくても良いが,場所の文化/自然等は変化し続けている とは}
      前半の内容については,
      \begin{center}
        場所のセンス$\Leftrightarrow $没場所性\\
        ※場所のセンスと没場所性が表裏一体
      \end{center}
      についてであり,場所として存在するにはどちらでも良いと思い, 上のように考える.\\
      \quad 一方後半についてであるが, これは場所が完全に変わらないということはあり得ないということである.\\
      \quad 例えば授業では,複数の北海道の開拓について述べられていた. 北海道は元々はアイヌの人々のものであったが,そこから開拓民がやってきたら稲作が行われ(これは社会的主観で行われた),
      また米国的な開拓も行われ(気候が似ているからということで行われた),現在ではこの2つの両方が存在する状況になっている.まさに文化を変化させていると思う.
      稲作も,最初はできなかったが突然変異で可能になったというのはまさしく自然(植えられた稲を自然というのか怪しいが)の変化になると思う.\\
      \quad また,産業革命以降のマンチェスター市の人口推移や工業の変化からも考えることはできると思う.
      \begin{figure}[htbp]
        \centering
        \includegraphics[height=35mm, width=70mm]{Manchester_Population2.png}
        \caption{19世紀のマンチェスター市の人口推移}
        \label{fig:steam_engine}
      \end{figure}\\
      参考文献\cite{population}によれば,産業革命前は人口が少ない街\footnote{同じ参考文献によると,
      1081年時点の人口は3000人であり,そもそも我々基準では街であるかも微妙ではある.}であったが,そこから人口が増え,その後は衰退しつつ少し回復している.
      これは,歴史の講義で得た知識を参考に考えると繊維産業が栄え人口が増加し,その後は繊維産業の衰退で人口が減少\footnote{途中途切れているのは,戦争で国勢調査が実施されていないためである.}
      という流れであると考えられるが,少なくともマンチェスターという場所は不変ではないことがわかる.\\
      \quad その他外来種問題\cite{env1},開国以前と開国後の文明開花した日本のような文化の変化などもあるが,結局のところ, \textbf{場所の文化/自然等というものはその時代を生きる人々に影響される}のではないだろうかと考える.
    
  \begin{thebibliography}{9}

% \bibitem{WebExample}
% 著者「Webサイト名」
% \newblock \url{https://www.y-history.net/appendix/wh1101-025.html}
% \newblock (2025.8.22閲覧).

% \bibitem{BookExample}
% 著者『Webサイト名』
% \newblock 出版社,出版年.

% \bibitem{BookExample}
% 著者『Webサイト名』
% \newblock 出版社,出版年.

\bibitem{yurucha}
富士宮市観光協会「『ゆるキャン△』特設サイト SEAZON2 7話」
\url{https://fujinomiya.gr.jp/yurucamp/season2-7/}
(2025.11.29閲覧).

\bibitem{population}
Manchester City Council「Public Intelligence A20 Manchester's population over time」
\url{https://www.manchester.gov.uk/download/downloads/id/25393/a20_1086-2016_manchester_population.pdf}
(2025.11.29閲覧).

\bibitem{env1}
環境省「外来種問題を考える」
\newblock \url{https://www.env.go.jp/nature/intro/2outline/index.html}
(2025.11.29閲覧).

  \end{thebibliography}
\end{document}